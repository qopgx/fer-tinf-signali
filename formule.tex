\documentclass{article}
\usepackage{amsmath}
\usepackage{amssymb}
\usepackage{xcolor}
% Title formatting
\usepackage{titling}
\setlength{\droptitle}{-7em}
\posttitle{\end{center}\vspace{-5em}}
\title{TINF Signali formule\linebreak \linebreak\normalsize \textcolor{red}{OVE FORMULE NE SADRŽAVAJU POSTUPKE RJEŠAVANJA ZADATAKA, BROJČANE REZULTATE NITI NAPUTKE ZA RJEŠAVANJE. ISKLJUČIVO SADRŽE ZNANJE IZ KNJIGE I ZBIRKE.}}
\author{}
\date{}
\begin{document}
\maketitle
\subsection*{Periodični signali}
Snaga:
$$
	P=\lim_{k\rightarrow\infty}[\frac{1}{kT_0}k\int_{0}^{T_0}\lvert x(t)\rvert^2dt]=\frac{1}{T_0}\int_{0}^{T_0}\lvert x(t)\rvert^2dt=\sum_{k=-\infty}^{\infty}\lvert c_k\rvert^2=...
$$

Sinusni:
$$
	P = \frac{A^2}{2}
$$
Još za slijed pravokutnih:
$$
	P=A^2\frac{\tau}{T}
$$
Fourier parovi:
$$
	x(t)=sin(\omega_0 t)\longleftrightarrow -j\frac{A}{2}[\delta(f-f_0) - \delta(f+f_0)]=X(f)
$$
$$
	x(t)=cos(\omega_0 t) \longleftrightarrow \frac{A}{2}[\delta(f-f_0) + \delta(f+f_0)]
$$
\subsection*{Neperiodični signali}
$\lvert X(f)\rvert$ je amplitudni spektar, $\theta(f)$ je fazni.
$$
	X(f)=\lvert X(f)\rvert e^{j\theta(f)}
$$
\subsubsection*{Parsevalov teorem}
$$
	E=\int_{-\infty}^{\infty}\lvert x(t)\rvert^2dt=\int_{-\infty}^{\infty}\lvert X(f)\rvert^2 df = \frac{1}{2\pi}\int_{-\infty}^{\infty}\lvert X(\omega)\rvert^2d\omega
$$
\subsubsection*{Pravokutni impuls}
$$
	P=0\quad\text{(beskonačnost)}, \quad E=A^2\tau
$$
\subsection*{Slučajni signali}

\subsubsection*{Pravila E}
\begin{itemize}
	\item $E[c]=c,\quad c\in\mathbb{R}$
	\item $E[cX]=cE[X]$
	\item $E[X+Y]=E[X]+E[Y]$
	\item $E[XY]=E[X]E[Y]$
\end{itemize}
\subsubsection*{Stacionarnost}
Uvjeti
\begin{itemize}
\item $E[X(t)]=\mu_x$
\item $\forall t_1,t_2,\quad R_x(t_1,t_2)=R_x(t_1,t_1+\tau)$
\subitem Pri tome: $R_x$ parna, $\lvert R_x \rvert \geq R_x(0) \geq 0$
\end {itemize}
Srednja snaga
$$
	P=E[X^2(t)]=R_X(0)=\int_{-\infty}^{\infty}S_X(f)df
$$
$$
	E[X]=0 \rightarrow P=var(X)=\sigma_X^2
$$
\subsubsection*{Bijeli šum}
$W(t)$ bijeli šum ako:
\begin{itemize}
	\item $R_W(\tau) = C_1*\delta(\tau)$
	\item $C_W(\tau) = C_2*\delta(\tau)$
\end{itemize}
Svojstva:
\begin{itemize}
	\item $\mu_W=0$
	\item $R_W(\tau)=\sigma^2\delta(\tau)=N_0/2$
	\item $S_W(f) = \sigma^2\int_{-\infty}^{\infty}\delta(t)e^{-j2\pi ft}dt=\sigma^2=N_0/2$
\end{itemize}
Gaussova razdioba:
$$
	f_x(x)=\frac{1}{\sigma_X\sqrt{2\pi}}e^{-(x-\mu_X)^2 / (2\sigma_X^2)}
$$
\subsection*{Prijenos}
Izlazni signal:
$$
	y(t)=\int_{-\infty}^{\infty}x(\tau)h(t-\tau)d\tau=\int_{-\infty}^{\infty}h(\tau)x(t-\tau)d\tau
$$
Prijenosna funkcija:
$$
	H(f)=\int_{-\infty}^{\infty}h(t)e^{-j2\pi ft}dt
$$
Amplitudni odziv RC:
$$
	20\log\frac{\lvert H(f) \rvert}{\lvert H(0) \rvert}=20\log(\lvert H(f) \rvert)
$$
\[
	\lvert H(f)\rvert=
	\begin{cases}
		1, \quad \text{za } \lvert f\rvert \le f_g \\
		0, \quad \text{za } \lvert f\rvert \ge f_g
	\end{cases}
\]
Impulsni odziv i prijenosna funkcija
$$
	y(t)=\int_{-\infty}^{\infty}x(\tau)h(t-\tau)d\tau=\int_{-\infty}^{\infty}h(\tau)x(t-\tau)
$$
$$
	y(t)=x(t)*h(t)
$$
$$
	Y(f)=X(f)*H(f)
$$
Amplitudni odziv je parna funkcija frekvencije, a fazni odziv neparna:
$$
	\lvert H(-f)\rvert=\lvert H(f)\rvert
$$
$$
	\theta(-f)=-\theta(f)
$$
Ako je $X(t)$ stacionarni slučajni proces...
$$
	\mu_Y=\mu_X H(0)
$$
$$
	S_Y (f)=S_X (f)\lvert H(f)\rvert^2
$$
Ako na ulazu kanala dovedemo signal $x(t)$ čiji je spektar $X(f)$ definiran kao $X(f)=\lvert X(f)\rvert e^{j\varphi(f)}$, onda $Y(f)$ zadovoljava:
$$
	Y(f)=\lvert Y(f)\rvert e^{j\vartheta(f)}
$$
$$
	\lvert Y(F)\rvert=\lvert X(f)\rvert\lvert H(f)\rvert
$$
$$
	\vartheta(f)=\varphi(f)-\theta(f)
$$
Amplitudni odziv RC:
$$
	\lvert H(f)\rvert=\lvert\frac{U_\text{izlaz}(f)}{U_\text{ulaz}(f)}\rvert=\frac{1}{\sqrt{1+(2\pi fRC)^2}}
$$
\subsection*{Uzorkovanje i kvantizacija}
Frekvencija uzorkovanja u pomaknutom pojasu:
$$
	f_u=2*\frac{B+B_0}{M+1},\quad M_m=\lfloor\frac{B_0}{B}+1 \rfloor
$$
Varijanca kvantizacijskog šuma tj. srednja snaga kvantizacijskog šuma:
$$
	var(Q)=\sigma_Q^2=\frac{\Delta^2}{12}=\frac{1}{3}m_{max}^2 2^{-2r},\quad \Delta=\frac{2m_{max}}{L}
$$
Omjer srednja snaga signala i snaga kvantizacijskog šuma:
$$
	\frac{S}{N}=\frac{S}{\sigma_Q^2}=(\frac{3S}{m_{max}^2})2^{2r}
$$
U decibelima, samo za sinusni:
$$
	(\frac{S}{N_q})_{dB}=1.76+6.02*r
$$
Brzina prijenosa:
$$
	R=f_u*r
$$
\subsubsection*{Entropija u K.K.}
$f$ su funkcije gustoće vjerojatnosti.
$$
	H(X)=E[-log(f_X(X))]=-\int_{-\infty}^{\infty}f_X(x)log(f_X(x))dx
$$
$$
	f_X(x)=\int_{-\infty}^{\infty}f(x,y)dy
$$
$$
	f_Y(x)=\int_{-\infty}^{\infty}f(x,y)dx
$$
$$
	H(X|Y)=E[-log(f_Y(X|Y))]=-\int_{-\infty}^{\infty}\int_{-\infty}^{\infty}f(x,y)log(\frac{f(x,y)}{f_Y(y)})dxdy
$$
$$
	H(X,Y)=E[-log(f(X,Y))]=-\int_{-\infty}^{\infty}\int_{-\infty}^{\infty}f(x,y)log(f(x,y))dxdy
$$
$$
	I(X;Y)=E[-log(f_Y(X|Y))]=-\int_{-\infty}^{\infty}\int_{-\infty}^{\infty}f(x,y)log(\frac{f(x,y)}{f_X(x)f_Y(y)})dxdy
$$
Prijenos u prisutstvu aditivnog šuma
$$
	f_x(y|x)=f_x(z+x|x)=f_z(z)
$$
$$
	I(X;Y)=H(Y)-H(Y|X)=H(Y)-H(Z)
$$
Kapacitet:
$$
	C=\textnormal{max }I(X;Y)=\textnormal{max }[\quad ln[\sqrt{2\pi e(\sigma_x^2+\sigma_y^2)}]\quad -\quad \frac{1}{2}ln(2\pi e\sigma_z^2) \quad]=...
$$
$$
	...=\frac{1}{2}ln(\frac{\sigma_x^2+\sigma_y^2}{\sigma_z^2})=\frac{1}{2}ln(1+\frac{S}{N})
$$
\subsubsection*{Maximizacija entropije u K.K.}
\begin{itemize}
	\item $x\in [a,b]\rightarrow f_1(x)=\frac{1}{b-a},\quad H(X)=ln(b-a) [\frac{\text{nat}}{\text{symb}}]$
	\item $x\ge 0\land E[X]=a,a\le 0\rightarrow f_1(x)=\frac{1}{a}e^{-\frac{x}{a}},\quad H(X)=ln(ae)=1+ln(a)$
	\item $E[X]=0\land\exists\sigma_X\rightarrow f\textnormal{ gauss},\quad H(X)=ln(\sigma_X\sqrt{2\pi e})$
\end{itemize}
\subsection*{Ostalo}
Srednja kvadratna pogreška, $u_{qi}$ kvantizacijske razine.
$$
	N_q^2=\sum_{u_{qi}}\int_{u_{qi} - \Delta/2}^{u_{qi} + \Delta/2}(u-u_{qi})^2f(u)du\quad[V^2]
$$
\subsection*{Konverzije}

U decibel (dB):
$$
	x\rightarrow10\log_{10}(x)
$$
\subsection*{Jedinice}
$$
	c_k \leftrightarrow [\frac{V}{Hz}]
$$
$$
	S_X(f) \leftrightarrow [\frac{W}{Hz}]
$$
\end{document}
