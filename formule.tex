\documentclass[9pt]{extarticle}
\usepackage{amsmath}
\usepackage{amssymb}
\usepackage{xcolor}
\usepackage{multicol}
\setlength{\columnsep}{50pt}
\usepackage{geometry}
\geometry{
    a4paper,
    margin=0.5in,
    includeheadfoot
}
\usepackage{titling}
\setlength{\droptitle}{-7em}
\posttitle{\par\end{center}\vspace{-5em}}
\title{TINF Signali formule}
\author{}
\date{}

\newcommand{\infintegral}{\int_{-\infty}^{\infty}}

\begin{document}
\maketitle
\begin{multicols}{2}
	\[
		E=I(Ri^2(t))=I(\frac{u^2(t)}{R}), \quad P=\lim_{T\to\infty}\frac{1}{T}\int_{-T_0/2}^{T_0/2}Ri^2(t)dt
	\]

	\section*{Periodični signali}
	Fourierov razvoj:
	\[
		x(t)=\sum_{k=-\infty}^{\infty}c_ke^{jk\omega_0t}, \text{ gdje } c_k = \frac{1}{T}\int_{-T_0/2}^{T_0/2}x(t)e^{-jk\omega_0t}dt
	\]

	\subsection*{Snaga}
	\begin{align*}
		P & = \lim_{k \to \infty} \left[ \frac{1}{kT_0} \int_{0}^{kT_0} |x(t)|^2 \, dt \right] \\
		  & = \frac{1}{T_0} \int_{0}^{T_0} |x(t)|^2 \, dt                                      \\
		  & = \sum_{k=-\infty}^{\infty} |c_k|^2
	\end{align*}

	Za sinusni signal:
	\[
		P = \frac{A^2}{2}
	\]

	Za slijed pravokutnih impulsa:
	\[
		P = A^2 \frac{\tau}{T}
	\]

	\subsection*{Fourierovi parovi}
	\[
		x(t) = \sin(\omega_0 t) \leftrightarrow -j \frac{A}{2} [\delta(f - f_0) - \delta(f + f_0)]
	\]

	\[
		x(t) = \cos(\omega_0 t) \leftrightarrow \frac{A}{2} [\delta(f - f_0) + \delta(f + f_0)]
	\]

	\section*{Neperiodični signali}
	Spektar
	\[
		X(f)=\infintegral x(t)e^{-j2\pi ft}dt, \quad x(t)=\infintegral X(f)e^{j2\pi ft}df
	\]

	$X(f) = |X(f)| e^{j \theta(f)}$, gdje je $|X(f)|$ amplitudni spektar, a $\theta(f)$ fazni spektar.

	\subsection*{Parsevalov teorem}
	\[
		E = \int_{-\infty}^{\infty} |x(t)|^2 \, dt = \int_{-\infty}^{\infty} |X(f)|^2 \, df = \frac{1}{2\pi} \int_{-\infty}^{\infty} |X(\omega)|^2 \, d\omega
	\]

	\subsection*{Pravokutni impuls}
	\[
		P = 0 \quad (\text{beskonačnost}), \quad E = A^2 \tau
	\]

	\section*{Slučajni signali}
	Srednja vrijednost
	\[
		\mu_X(t)=E\left[X(t)\right]=\infintegral xf_X(x,t)dx
	\]
	Autokorelacijska funkcija
	\[
		R_X(t_1,t_2)=E\left[X(t_1)X(t_2)\right]
	\]
	Autokovarijanca
	\begin{align*}
		C_X(t_1,t_2)=E\left\{\left[X(t_1) - \mu_x(t_1)\right]\left[X(t_2) - \mu_x(t_2)\right]\right\}
	\end{align*}

	\subsection*{Pravila očekivanja (E)}
	\[
		E[c] = c, \quad c \in \mathbb{R}, \quad E[cX] = c E[X]
	\]
	\[
		E[X + Y] = E[X] + E[Y], \quad E[XY] = E[X] E[Y]
	\]

	\subsection*{Stacionarnost}
	Uvjeti:
	\begin{itemize}
		\item $E[X(t)] = \mu_x$
		\item $\forall t_1, t_2, \quad R_x(t_1, t_2) = R_x(t_1 - t_2) = R_x(\tau)$
		      \subitem Pri tome: $R_x$ je parna funkcija, $|R_x(\tau)| \leq R_x(0) \geq 0$
	\end{itemize}

	Srednja snaga:
	\[
		P = E[X^2(t)] = R_X(0) = \int_{-\infty}^{\infty} S_X(f) \, df
	\]
	\[
		E[X]=0 \longrightarrow P = \operatorname{var}(X) = \sigma_X^2
	\]
	\subsection*{Bijeli šum}
	$W(t)$ je bijeli šum ako:
	\[
		R_W(\tau) = C_1 \delta(\tau) \quad\land\quad C_W(\tau) = C_2 \delta(\tau)
	\]

	Svojstva:
	\begin{itemize}
		\item $\mu_W = 0$
		\item $R_W(\tau) = \sigma^2 \delta(\tau) = N_0 / 2$
		\item $S_W(f) = \sigma^2 \int_{-\infty}^{\infty} \delta(t) e^{-j 2\pi f t} \, dt = \sigma^2 = N_0 / 2$
	\end{itemize}

	Gaussova razdioba:
	\[
		f_x(x) = \frac{1}{\sigma_X \sqrt{2\pi}} e^{-(x - \mu_X)^2 / (2 \sigma_X^2)}
	\]
	\section*{Prijenos}

	Izlazni signal:
	\[
		y(t) = \int_{-\infty}^{\infty} x(\tau) h(t - \tau) \, d\tau = \int_{-\infty}^{\infty} h(\tau) x(t - \tau) \, d\tau
	\]

	Prijenosna funkcija:
	\[
		H(f) = \int_{-\infty}^{\infty} h(t) e^{-j 2\pi f t} \, dt
	\]

	Amplitudni odziv RC kruga:
	\[
		20 \log \frac{|H(f)|}{|H(0)|} = 20 \log |H(f)|
	\]

	Za idealni filtar:
	\[
		|H(f)| =
		\begin{cases}
			1, & |f| \leq f_g \\
			0, & |f| > f_g
		\end{cases}
	\]

	Impulsni odziv i prijenosna funkcija:
	\[
		y(t) = x(t) * h(t), \quad Y(f) = X(f) H(f)
	\]

	Amplitudni odziv je parna funkcija, a fazni neparna:
	\[
		|H(-f)| = |H(f)|, \quad \theta(-f) = -\theta(f)
	\]

	Ako je $X(t)$ stacionarni slučajni proces:
	\[
		\mu_Y = \mu_X H(0), \quad S_Y(f) = S_X(f) |H(f)|^2
	\]

	Ako je ulaz $x(t)$ sa spektrom $X(f) = |X(f)| e^{j \varphi(f)}$:
	\[
		Y(f) = |Y(f)| e^{j \vartheta(f)}, \quad |Y(f)| = |X(f)| |H(f)|
	\]
	\[
		\vartheta(f) = \varphi(f) + \theta(f)
	\]

	Amplitudni odziv RC kruga:
	\[
		|H(f)| = \left| \frac{U_{\text{izlaz}}(f)}{U_{\text{ulaz}}(f)} \right| = \frac{1}{\sqrt{1 + (2\pi f RC)^2}}
	\]

	\section*{Uzorkovanje i kvantizacija}

	Frekvencija uzorkovanja u pomaknutom pojasu:
	\[
		f_u = 2 \frac{B + B_0}{M + 1}, \quad M_m = \left\lfloor \frac{B_0}{B} + 1 \right\rfloor
	\]

	Varijanca kvantizacijskog šuma (srednja snaga):
	\[
		\operatorname{var}(Q) = \sigma_Q^2 = \frac{\Delta^2}{12} = \frac{1}{3} m_{\max}^2 2^{-2r}, \quad \Delta = \frac{2 m_{\max}}{L}
	\]

	Omjer srednje snage signala i snage kvantizacijskog šuma:
	\[
		\frac{S}{N} = \frac{S}{\sigma_Q^2} = \left( \frac{3S}{m_{\max}^2} \right) 2^{2r}
	\]

	U decibelima (samo za sinusni signal):
	\[
		\left( \frac{S}{N_q} \right)_{dB} = 1.76 + 6.02 r
	\]

	Brzina prijenosa:
	\[
		R = f_u r \quad \left[ \frac{\text{bit}}{s} \right]
	\]
	\subsection*{Entropija u kontinuiranom kanalu}
	$f$ su funkcije gustoće vjerojatnosti.

	\[
		H(X) = E[-\log f_X(X)] = -\int_{-\infty}^{\infty} f_X(x) \log f_X(x) \, dx
	\]
	\[
		f_X(x) = \int_{-\infty}^{\infty} f(x,y) \, dy, \quad f_Y(y) = \int_{-\infty}^{\infty} f(x,y) \, dx
	\]

	\begin{align*}
		H(X|Y) & = E[-\log f_{X|Y}(X|Y)]                                                                                         \\
		       & = -\int_{-\infty}^{\infty} \int_{-\infty}^{\infty} f(x,y) \log \left( \frac{f(x,y)}{f_Y(y)} \right) \, dx \, dy
	\end{align*}
	\begin{align*}
		H(X,Y) & = E[-\log f(X,Y)]                                                                 \\
		       & = -\int_{-\infty}^{\infty} \int_{-\infty}^{\infty} f(x,y) \log f(x,y) \, dx \, dy
	\end{align*}
	\begin{align*}
		I(X;Y) & = E[-\log f_{Y|X}(Y|X)]                                                                                               \\
		       & = \int_{-\infty}^{\infty} \int_{-\infty}^{\infty} f(x,y) \log \left( \frac{f(x,y)}{f_X(x) f_Y(y)} \right) \, dx \, dy
	\end{align*}
	Prijenos u prisutnosti aditivnog šuma:
	\[
		f_x(y|x) = f_x(z+x|x) = \phi(z)
	\]
	\[
		I(X;Y) = H(Y) - H(Y|X) = H(Y) - H(Z)
	\]

	Kapacitet:
	\begin{align*}
		C & = \max I(X;Y) = \max \left[ \frac{1}{2} \ln [2\pi e (\sigma_X^2 + \sigma_Z^2)] - \frac{1}{2} \ln (2\pi e \sigma_Z^2) \right] \\
		  & = \frac{1}{2} \ln \left( 1 + \frac{S}{N} \right) \quad \left[ \frac{\text{nat}}{s} \right]
	\end{align*}

	\[
		C = \frac{1}{2} \log_2 \left( 1 + \frac{S}{N} \right) \quad [\text{bit/simbol}]
	\]

	\subsection*{Maksimizacija entropije u kontinuiranom kanalu}
	\begin{itemize}
		\item $x \in [a,b] \to f(x) = \frac{1}{b-a}$, $\quad H(X) = \ln(b-a) \quad [\text{nat/sym}]$
		\item $x \geq 0 \land E[X] = a > 0 \to f(x) = \frac{1}{a} e^{-x/a}$, $\quad H(X) = \ln(a e) = 1 + \ln a$
		\item $E[X] = 0 \land \exists \sigma_X \to f$ Gaussova, $\quad H(X) = \ln(\sigma_X \sqrt{2\pi e})$
	\end{itemize}

	\subsection*{Inf. kapacitet AWGN kanala}
	Za kanal s $f_u=2B$...
	\[
		n=2B\longrightarrow B\log_2{\left(1+\frac{S}{N} \right)} \quad \left[bit/s\right]
	\]
	\[
		C=2BD
	\]
	$E_b$, srednja energija po svakom bitu...
	\[
		\text{uz... } E_b=S/R_b, \quad S=E_bC, \quad \frac{C}{B}=\log_2\left(1 + \frac{E_b}{N_0}\frac{C}{B} \right)
	\]
	\[
		\frac{E_b}{N_0}=\frac{2^{C/B} - 1}{C/B}, \quad \lim_{B\to\infty}\left(\frac{E_b}{N_0}\right)=\log\left(2\right), \quad \lim_{B\to\infty}C=\frac{S}{N_0}\log_2e
	\]
	\section*{Ostalo}

	Srednja kvadratna pogreška, $u_{qi}$ kvantizacijske razine:
	\[
		N_q^2 = \sum_{u_{qi}} \int_{u_{qi} - \Delta/2}^{u_{qi} + \Delta/2} (u - u_{qi})^2 f(u) \, du \quad [V^2]
	\]

	\section*{Konverzije}
	Pojačanje. U decibele (dB): $x \to 10 \log_{10} (x)$
	\section*{Jedinice}
	\[
		c_k \leftrightarrow \left[ \frac{V}{Hz} \right], \quad S_X(f) \leftrightarrow \left[ \frac{W}{Hz} \right]
	\]
	\newpage
	\subsection*{Entropija slučajnog vektora}
	\begin{align*}
		H(\mathbf{X}) & =E\left[ -\log\left\{ X_1,...,X_n \right\} \right]                                                            \\
		              & = -\infintegral...\infintegral f_\mathbf{X}(x_1,...,x_n)\log\left[f_\mathbf{X}(x_1,...,x_n)\right]dx_1...dx_n
	\end{align*}
	\subsection*{Inf. kapacitet AWGN kanala}
	Pri uzorkovanju:
	\[
		\mathbf{X}=\left[X_1,X_2,...,X_n\right]
	\]
	\[
		\mathbf{Y}=\mathbf{X}+\mathbf{Z}
	\]
	\[
		E[X_k]=0, \quad E[X_k^2]=\sigma_{xk^2}
	\]
	\[
		\phi(\mathbf{z})=\prod_{k=1}^{n}\left[\frac{1}{\sigma_{z_k}\sqrt{2*\pi}}e^{-z_k^2/2\sigma_{z_k}^2} \right]
	\]
	\[
		H(\mathbf{Y}|\mathbf{X})=H(\mathbf{Z})=-\infintegral\phi(\mathbf{z})\log\left[ \phi(\mathbf{z}) \right]=\sum_{k=1}^n\log(\sigma_{z_k}\sqrt{2\pi e})
	\]
	\[
		I(\mathbf{X};\mathbf{Y})=H(\mathbf{Y})-\sum_{k=1}^n\log(\sigma_{z_k}\sqrt{2\pi e})
	\]
	Ako su sve varijance jednake...
	\begin{align*}
		I_{\text{max}}(\mathbf{X};\mathbf{Y}) & =\frac{n}{2}\log\left(1+\frac{\sigma_x^2}{\sigma_z^2}\right) \quad \left[bit/simbol\right] \\
		                                      & =\frac{n}{2}\log\left(1+\frac{S}{N}\right)
	\end{align*}

\end{multicols}
\end{document}

